\section{Introduzione generale}



\subsection{Esercizio assegnato}
Per il superamento della parte di Laboratorio di Algoritmi e Strutture Dati ho svolto il seguente esercizio:

\begin{itemize}
    \item B : Statistiche d'ordine dinamiche - varie implementazioni
\end{itemize}


\subsection{Breve descrizione dello svolgimento degli esercizi}
Per ogni esercizio suddivideremo la sua descrizione in 4 parti fondamentali:

\begin{itemize}
    \item \textbf{Spiegazione teorica del problema} : qui è dove si descrive il problema che andremo ad affrontare in modo teorico partendo dagli assunti del libro di Algoritmi e Strutture Dati e da altre fonti.
    \item \textbf{Documentazione del codice} : in questa parte spieghiamo come il codice dell'esercizio viene implementato 
    \item \textbf{Descrizione degli esperimenti condotti} : partendo dal codice ed effettuando misurazioni varie cerchiamo di verificare le ipotesi teoriche
    \item \textbf{Analisi dei risultati sperimentali} : dopo aver svolto i vari esperimenti riflettiamo sui vari risultati ed esponiamo una tesi
\end{itemize}

\subsection{Specifiche della piattaforma di test}
La piattaforma utilizzata per i test è un Laptop Lenovo, modello ideapad 700-15ISK con Sistema operativo Windows 10 Home versione 21H2, 64 bit. L'Hardware del dispositivo è il seguente:

\begin{itemize}
    \item \textbf{CPU} : Intel Core i7-6700HQ CPU 2.60GHz   
    \item \textbf{RAM} : 8GB
    \item \textbf{SSD} : 465GB 
\end{itemize}

Il linguaggio di programmazione utilizzato sarà Python e la piattaforma in cui il codice è stato scritto ed eseguito è l'IDE \textbf{Spyder}. 
La stesura di questo testo è avvenuta tramite l'utilizzo dell'editor online \textbf{Overleaf}.
